% Podstawowe definicje dla wszystkich dokumentów

\documentclass[12pt]{mwart}

\usepackage[OT4,plmath]{polski}
\usepackage{amsmath,amssymb,amsfonts,amsthm,mathtools}
\usepackage{color}
\usepackage{fontspec}
\usepackage{listings,times}

\usepackage{bbm}
\usepackage[colorlinks=true, urlcolor=blue, linkcolor=black]{hyperref}
\usepackage{url}
\usepackage{graphicx}

\graphicspath{{images/}}

\newcommand{\HRule}{\rule{\linewidth}{0.5mm}}

\usepackage{multicol}

\usepackage{lmodern} \normalfont

\widowpenalty=10000
\clubpenalty=10000
\raggedbottom

\usepackage{caption}

\setcounter{secnumdepth}{5}

\begin{document}
\begin{centering}
  {\Large \textbf{Sprawozdanie}}\\
  Autorzy: Łukasz Czapliński, Diana Czepirska \\ 
  Kierunek: Informatyka Rok: 3 Rok akademicki: 2013/2014 Semestr: zimowy
  {\large \textbf{Ćwiczenie nr 26:  Bramki i układy logiczne}}\\
  Prowadzący ćwiczenia: dr Radosław Wasilewski \\
  Grupa: 1; wtorek 16:00 - 19:00 \\
  Data: 2013-12-17 \\
\end{centering}
\section{Wstęp teoretyczny}
Komputery działają użwając jedynie dwóch cyfr $0$ i $1$. Dzięki temu budowa komputerów może być stosunkowo prosta: operują na dwóch stanach (wysokim oraz niskim), a wszystkie operacje są odpowiednikami działań w logice Boole'a.\\
Prowadzi to jednak do problemu: jak reprezentować liczby? Człowiek przyzwyczaił się do liczenia używając liczb naturalnych (potem całkowitych, rzeczywistych itd.) oraz systemu dziesiętnego. Jak nauczyć tego komputer? Rozważymy tutaj problem liczb naturalnych.
\subsection{Kod Graya}
Oczywiste jest, że do reprezentacji każdej liczby potrzebna będzie sekwencja wartości $0$ i $1$. Pozostaje rozważyć, jaka sekwencja ma odpowiadać jakiej liczbie. Kod jest takim właśnie mapowaniem: z pewnego podzbioru liczb naturalnych (ciągu następujących po sobie liczb, zaczynającego się od 0) na sekwencję $0$ i $1$.\\
Najbardziej znanym kodem tego typu jest kod binarny. Reprezentacja liczby w tym kodzie to ciąg cyfr ($0$ i $1$), z których każda jest mnożnikiem kolejnych potęg dwójki (tak jak w systemie dziesiętnym - każda cyfra jest mnożnikiem kolejnych potęg dziesiątki).\\
Kodowanie binarne nie jest jednak odporne na błędy. Pomyłka podczas transmisji jednego bitu może powodować błąd względny tym większy, im starszy był to bit. O wiele lepiej byłoby jeśli kolejne liczby różniłyby się tylko jednym bitem. Wówczas pomyłka w transmisji zwykle powoduje o wiele mniejsze błędy. Kod Graya jest właśnie takim kodowaniem. Przykład dla 4 bitów: Tabela~\ref{T2}.

\begin{table}[hbt!]
\begin{center}
\begin{tabular}{r | l}
  Liczba  & Kod \\
  \hline
		0	&	0000 \\
		1 & 0001 \\
		2 & 0011 \\
		3 & 0010 \\
		4 & 0110 \\
		5 & 0111 \\
		6 & 0101 \\
		7 & 0100 \\
		8 & 1100 \\
		9 & 1101 \\
		10 & 1111 \\
		11 & 1110 \\
		12 & 1010 \\
		13 & 1011 \\
		14 & 1001 \\
		15 & 1000 \\
  \hline
  \multicolumn{2}{|c|}{itd.} \\
  \hline
\end{tabular}
\caption{Kod Graya dla sekwencji 4-bitowej}
\label{T2}
\end{center}
\end{table}

\subsection{Kontrola parzystości}
Odporność na błędy jest kluczowym problemem w systemach komputerowych. Warto więc rozważyć układy mające na celu znajdywanie i ewentualną korekcję błędów. Najprostszym (i często stosowanym w złożonych systemach) rozwiązaniem jest kontrola przystości. Zlicza się ilość $1$ modulo 2 w danym ciągu. Wówczas można sprawdzić, czy w przesyłanym ciągu wystąpiła nieparzysta ilość błędów -- pozwala to na uniknięcie połowy błędów. \\
\indent W tym celu używa się bramki XOR. Spełnia ona następującą funkcję: Tabela~\ref{T3} i~ma następujący symbol: Rysunek~\ref{F1}.
\begin{table}[hbt!]
\begin{center}
  \begin{tabular}{c | c | c}
    y|x & 0 & 1 \\
    \hline
    0   & 0 & 1 \\ 
    1   & 1 & 0 \\
  \end{tabular}
  \caption{Tabela wartości działania XOR}
  \label{T3}
\end{center}
\end{table}
\begin{figure}[hbt!]
\begin{center}
  \includegraphics{XOR.png}
  \caption{Symbol bramki XOR}
  \label{F1}
\end{center}
\end{figure}
\subsection{Układy kombinacyjne}
Układy kombinacyjne to rodzaj układów cyfrowych (złożonych z bramek logicznych), w których stan wyjść zależy wyłącznie od stanu wejść (inaczej niż w przypadku układów sekwencyjnych - tam stan wyjść zależy od stanu wejść oraz od poprzedniego stanu wyjść). Stan wyjść w układach kombinacyjnych można opisać za pomocą funkcji boolowskich (w przeciwieństwie do układów sekwencyjnych). W układach tych nie występuje sprzężenie zwrotne.

\subsection{Zjawisko hazardu}
Jest to zjawisko występujące w układach cyfrowych, spowodowane niezerowym czasem propagacji sygnałów przez bramki logiczne. Zjawisko to jest niekorzystne, może bowiem spowodować błędne stany na wyjściach układów cyfrowych.

\section{Doświadczenia i obserwacje}
\subsection{Układ zamieniający kod dziesiętny na kod Graya}
\subsubsection{Cel}
Wykonanie odpowiedniego układu, nauka lutowania.
\subsubsection{Przebieg}
\begin{figure}[hbt!]
  \includegraphics[width=\textwidth]{gray.png}
  \caption{Układ zamieniający cyfry dziesiętne na kod Graya}
  \label{F2}
\end{figure}
\begin{enumerate}
  \item Zlutowaliśmy układ wg schematu (Rysunek~\ref{F2}).
	\item Przetestowaliśmy działanie układu dla cyfr $0$-$9$.
  \item Na podstawie wykonanych testów zapisaliśmy tabelę kodowania (Tabela~\ref{T4}).
	
  \begin{table}[hbt!]
	\begin{center}
  \begin{tabular}{r | l}
    Cyfra   & Kod \\
    \hline
    0 & 0000 \\  
		1 & 0001 \\ 
		2 & 0011 \\  
		3 & 0010 \\  
		4 & 0110 \\
    5 & 0111 \\  
		6 & 0101 \\
		7 & 0100 \\ 
		8 & 1100 \\ 
		9 & 1101 \\
  \end{tabular}
  \caption{Zamiana cyfr dziesiętnych na kod Graya}
  \label{T4}
	\end{center}
  \end{table}
\end{enumerate}
\subsubsection{Wnioski}
\begin{enumerate}
  \item Kod Graya jest bardziej odporny na błędy niż kod binarny (złe wykonanie jednego lutu zmieniało wynik tylko o 1).
	\item Kod Graya można rozszerzać podobnie jak binarny - dodając $0$ z przodu. 2 to $11$, $0011$, $0000011$ itd.
  \item Zamianę kodowania można wykonać jedynie diodami.
\end{enumerate}
\subsection{Układy logiczne}
\subsubsection{Cel}
Wykonać układ kontroli parzystości, zbudować oraz odkryć zastosowanie układów kombinacyjnych (Rysunek~\ref{F3}), zmierzyć czas propagacji sygnału przez bramki NOT i NAND.

\subsubsection{Przebieg}
\paragraph{Kontrola}\textbf{parzystości}\\
\begin{enumerate}
	\item Zaprojektowaliśmy układ kontroli parzystości słowa 8-bitowego za pomocą bramek XOR (Rysunek ~\ref{F6}).
	\item Zbudowaliśmy zaprojektowany układ korzystając z zestawu UNILOG.
	\item Przetestowaliśmy działanie układu, sporządziliśmy tabelę prawdy: Tabela ~\ref{T7}
\end{enumerate}
\begin{figure}[hbt!]
  \includegraphics[width=\textwidth]{konP.png}
  \caption{Układ kontroli parzystości słowa 4-bitowego}
  \label{F6}
\end{figure}
\begin{table}[hbt!]
\begin{center}
  \begin{tabular}{l | c}
    A\hspace{2 mm}B\hspace{2 mm}C\hspace{2 mm}D\hspace{2 mm}E\hspace{2 mm}F\hspace{2 mm}G\hspace{2 mm}H & Q \\
    \hline
    0\hspace{3 mm}0\hspace{3 mm}0\hspace{3 mm}0\hspace{3 mm}0\hspace{3 mm}0\hspace{3 mm}0\hspace{3 mm}0\hspace{3 mm} & 1\\ 
    0\hspace{3 mm}0\hspace{3 mm}1\hspace{3 mm}0\hspace{3 mm}0\hspace{3 mm}1\hspace{3 mm}1\hspace{3 mm}0\hspace{3 mm} & 0\\ 
		1\hspace{3 mm}1\hspace{3 mm}1\hspace{3 mm}1\hspace{3 mm}1\hspace{3 mm}1\hspace{3 mm}1\hspace{3 mm}1\hspace{3 mm} & 1\\
		0\hspace{3 mm}1\hspace{3 mm}0\hspace{3 mm}1\hspace{3 mm}0\hspace{3 mm}1\hspace{3 mm}0\hspace{3 mm}1\hspace{3 mm} & 1\\ 
		0\hspace{3 mm}0\hspace{3 mm}0\hspace{3 mm}1\hspace{3 mm}0\hspace{3 mm}0\hspace{3 mm}0\hspace{3 mm}0\hspace{3 mm} & 0\\ 
		1\hspace{3 mm}1\hspace{3 mm}1\hspace{3 mm}1\hspace{3 mm}0\hspace{3 mm}0\hspace{3 mm}0\hspace{3 mm}0\hspace{3 mm} & 1\\ 
  \end{tabular}
  \caption{Tabela prawdy układu kontroli parzystości}
  \label{T7}
\end{center}
\end{table}
\paragraph{Układy}\textbf{kombinacyjne}\\
\begin{enumerate}
	\item Zbudowaliśmy układy kombinacyjne na podstawie schematu: Rysunek~\ref{F3}
	\item Wyznaczyliśmy eksperymantalnie tabele prawdy obu układów: Tabela~\ref{T5}, Tabela~\ref{T6}
\end{enumerate}
\begin{figure}[hbt!]
  \includegraphics[width=\textwidth]{komb.png}
  \caption{Układy kombinacyjne}
  \label{F3}
\end{figure}
\begin{table}[hbt!]
\begin{center}
  \begin{tabular}{l | c}
    A\hspace{2 mm}B\hspace{2 mm} & Q \\
    \hline
    0\hspace{3 mm}0\hspace{3 mm} & 0\\
		0\hspace{3 mm}1\hspace{3 mm} & 1\\
		1\hspace{3 mm}0\hspace{3 mm} & 1\\
		1\hspace{3 mm}1\hspace{3 mm} & 0\\
  \end{tabular}
  \caption{Tabela prawdy układu kombinacyjnego a}
  \label{T5}
\end{center}
\end{table}
\begin{table}[hbt!]
\begin{center}
  \begin{tabular}{l | c}
    A\hspace{2 mm}B\hspace{2 mm} & Q \\
    \hline
    0\hspace{3 mm}0\hspace{3 mm} & 0\\
		0\hspace{3 mm}1\hspace{3 mm} & 1\\
		1\hspace{3 mm}0\hspace{3 mm} & 1\\
		1\hspace{3 mm}1\hspace{3 mm} & 0\\
  \end{tabular}
  \caption{Tabela prawdy układu kombinacyjnego b}
  \label{T6}
\end{center}
\end{table}
\paragraph{Czas}\textbf{propagacji}\\
\begin{enumerate}
	\item Zmierzyliśmy czas propagacji sygnału (za pomocą oscyloskopu) przez 4 szeregowo połączone bramki NOT (UCY 7404). Uzyskany wynik (czas propagacji przez jedną bramkę): \textbf{$8,75 ns$}
	
	\item Zmierzyliśmy czas propagacji sygnału (za pomocą oscyloskopu) przez 4 szeregowo połączone bramki NAND (UCY 7400). Uzyskany wynik (czas propagacji przez jedną bramkę): \textbf{$10 ns$}
\end{enumerate}
\begin{figure}[hbt!]
  \includegraphics[width=\textwidth]{notdel.png}
  \caption{Układy mierzący czas propagacji przez negatory}
  \label{F4}
\end{figure}
\begin{figure}[hbt!]
  \includegraphics[width=\textwidth]{nandel.jpg}
  \caption{Układy mierzący czas propagacji przez bramki NAND}
  \label{F5}
\end{figure}
\subsubsection{Wnioski}
\begin{enumerate}
  \item Układ kontroli parzystości można zbudować jedynie z bramek XOR..
	\item Zbudowane przez nas zgodnie ze schematem układy kombinacyjne okazały się być odpowiednikami bramki logicznej XOR - mają identyczne tabele prawdy.
  \item Czas propagacji sygnału przez bramkę NAND jest większy, niż przez bramkę NOT -- nie wystarczy pilnować, żeby wszystkie ścieżki w układzie miały tyle samo bramek logicznych: wystąpi hazard, nawet wówczas gdy jedna będzie miała dużo bramek jednego rodzaju, a druga identyczną ilość bramek innego.
\end{enumerate}
\end{document}
