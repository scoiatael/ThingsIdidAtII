\documentclass[11pt,wide]{mwart}

\usepackage[OT4,plmath]{polski}
\usepackage{amsmath,amssymb,amsfonts,amsthm,mathtools}
\usepackage{color}
\usepackage{fontspec}
\usepackage{listings,times}

\usepackage[colorlinks=true, urlcolor=blue]{hyperref}
\usepackage{url}

\setmainfont{Calibri Light}

\title{Sprawzdanie z zadania rozruchowego}
\author{Łukasz Czapliński}
\date{\today}

\begin{document}
\maketitle
\section{Temat}
Jako temat swojej pracy wybrałem grę w popularną grę karcianą Texas Hold'em. Nie jest ona skomplikowana: w każdym swoim ruchu gracz wybiera jeden z 3 ruchów: fold, pass lub raise. 
Jej zasady pokrótce opisałem na końcu dokumentu.
Opierałem się na wiedzy własnej.
\section{Przypadki testowe}
Pierwszy przypadek: dobry gracz, kiepskie karty na ręce, na stole w miarę niezłe karty.
Wydaje się naturalnym, żeby teraz blefować: inni nie widzą jego kart.\\
W czasie działania program dodatkowo pyta, czy gracz czuje się dobrze blefując (odpowiedź brzmi tak) oraz czy ma większe doświadczenie niż jego przeciwnicy (odpowiedź negatywna - gra porównywalnie długo).\\
Wyrok systemu brzmi: \textbf{pass}.\\
Faktycznie, skoro inni gracze grają porównywalnie długo, to mogliby przejrzeć taki blef.\\
\indent Drugi przypadek: kiepski gracz, dobre karty na ręce, kiepskie karty na stole. Wydaje się nieźle, powinien zrobić raise lub pass.\\
Program dopytuje się, czy możliwe jest, że pozostali gracze też będą blefować. Dodatkowo pyta teź o doświadczenie w pokerze. Otrzymuje odpowiedzi kolejno twierdzącą oraz kwalifikującą doświadczenie jako znikome.\\
Wynik: \textbf{raise}.\\
Nie jest to może idealna odpowiedź: bardziej pasowałby pass (jeśli ja bym grał). Jednakże agresywana gra może być tutaj dobrym rozwiązaniem: pozostali nie powinni się spodziewać takiego zagrania po mało doświadczonym graczu.\\
\section{Ocena i wnioski}
Jak wynika z podsumowania przypadków testowych, trudno jednoznacznie ocenić trafność odpowiedzi: należałoby go wnikliwie przetestować w praktyce. Jeśli okazałoby się, że jest nieźle to można nawet na takim programie zarabiać grając przez internet w pokera - najlepiej w innym kraju, gdzie hazard nie jest zakazany.\\
Prace rozwojowe powinny celować w rozgrywanie pełnych gier - łączyć odpowiedzi z kolejnych rozdań, by poprawiać swoje wyniki oraz by były bardziej konsekwentne. 
\pagebreak
\section{Zasady Texas Hold'em}
Na stole znajduje się pewna pula pieniędzy do której każdy z graczy dokłada. Zabiera ją gracz, który wygra dane rozdanie. Każdy gracz ma 2 karty. Dodatkowo na stole pojawiąją się karty - do 4. Celem jest uzyskanie jak najlepszej kombinacji kart swoich z tymi na stole.
\begin{itemize}
  \item [\textbf{Ruchy gracza}]
  \item fold - gracz porzuca to rozdanie - przestaje się w nim liczyć.
  \item pass - gracz nie zwiększa stawki - wyrównuje do obecnej i gra dalej.
  \item raise - gracz zwiększa stawkę - pozostali muszą wyrównać.
\end{itemize}
\end{document}

